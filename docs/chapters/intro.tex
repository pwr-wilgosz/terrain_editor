\chapter{Cel i zakres projektu}
\section{Zakres projektu}
Niniejszy projekt będzie zawierał opis techniczny i specyfikację kroków, wykorzystania narzędzi niezbędnych do stworzenia aplikacji w środowisku \textbf{Java IEE} .
Projekt będzie uwzględniał następujące etapy : 
\begin{enumerate}
\item{Analiza i Specyfikacja}
\item{Projektowanie}
\item{Implementacja}
\item{Testy}
\item{Ocena i Optymalizacja}
\end{enumerate}

\section{Cel projektu}
Celem projektu jest wykonanie aplikacji serwerowej. 
Aplikacja będzie budowana w oparciu o technologię \textbf{Java IEE}.
Sam sposób wdrożenia programu jak i jego zadania wraz z implementacją będą przebiegały wedle przedstawionego pomysłu, który został wstępnie zaakceptowany przez prowadzącego kurs.
Projekt ma na celu zapoznać Studentów specjalizacji Inżynieria Internetowa z procesem 
budowania oprogramowania obecnie stosowanego w świecie aplikacji serwerowych.
Projekt dodatkowo dzięki wybranemu tematowi pomoże zgłębić technologie strony klienckiej
takie jak \textbf{WebGL} i \textbf{JavaScript}

\chapter{Analiza i Specyfikacja}
\section{Opis słowny zadania}
Celem naszej pracy projektowej jest stworzenie aplikacji zdolnej do tworzenia i zarządzania mapkami 3d terenu.
Niniejsza aplikacja serwerowa ma za zadanie wspomagać tworzenia ciekawych wizualizacji terenu dla osób zajmujących się hobbystycznie jak i zawodowo kartografią.
Tworzone przez użytkowników prace można będzie łączyć, zlecać wykonanie na nich danych obliczeń w celu wykorzystania ich w danym problemie.
Nasz serwis z aplikacją nazwaliśmy \\\textbf{\textit{"Avalanche"}} \textit{[eng. lawina]}, w dalszej części projektowania i założeń podamy bliższe szczegóły naszych kroków projektowych.
Aplikację w przyszłości może uda rozwinąć się do dużo ciekawszych zastosowań.
\section{Specyfikacja wymagań funkcjonalnych}
\boxy{Prezentacja bazy punktów w formie mapy}{Jest to główna funkcja tej aplikacji, ma ona za zadnie z zadanego zbioru danych generować podgląd danego terenu w formie obrotowej mapki w przestrzeni trójwymiarowej.}
\newpage
\boxy{Oddzielne sesje dla każdego użytkownika}{Ponieważ aplikacja będzie zawierała w sobie proces tworzenia jakiegoś elementu terenu niezbędne będzie wyodrębnienie pojedynczych działań na aplikacji w formie sesji użytkowników, którzy swoje gotowe mapki będą mogli dzięki powiązaniu z kontem przechowywać w zdalnej przestrzeni dyskowej} 
\boxy{Wczytywanie mapek z pliku}{Poza wczytywaniem mapek z bazy serwera możliwe będzie wczytywanie wcześniej zapisanych mapek z pliku. Funkcjonalność taka będzie przydatna gdy użytkownik po wykasowaniu maki bądź usunięciu konta, chciałby odtworzyć swoje prace}
\boxy{Łączenie kilku prac w jedną}{Serwer będzie mógł według kilku definiowanych zasad łączyć kilka zasobów w jedną wspólną mapę, taka funkcjonalność będzie przydatna przy tworzeniu pracy opartej na wkładzie kilku użytkowników}
\boxy{Generacja map}{Generacja mapek na podstawie już istniejącej z podanymi zasadami zmiany oraz tworzenie totalnie losowej mapy}
\boxy{Zapisywanie prac na serwerze}{Każda stworzona mapka będzie dostępna niezależnie od miejsca i sprzętu użytkownika poprzez interfejs webowy dostępny w przeglądarce internetowej}
\boxy{Eksport gotowych prac do określonych formatów}{Każdą z gotowych prac będzie można wyeksportować do formatu możliwego do użytku w celu prezentacji/wizualizacji z założenia są to formaty 
\textbf{*.pdf *.jpeg}.}


\section{Specyfikacja wymagań niefunkcjonalnych}
\setcounter{boxCount}{1}
\boxy{Przejrzysty interfejs webowy}{Wymaganiem jest by aplikacja była nie przeładowana dodatkami i prosta w obsłudze dla osób nie posiadających zdolności programistycznych}
\boxy{Szybkość i oszczędność łącza}{Aplikacja powinna większość pracy wykonywać bez potrzeby generowania zbędnego ruchu sieciowego co pomoże zaoszczędzić zasoby klienta}
\boxy{Multiplatformowość}{Aplikacja powinna generować taki sam rezultat niezależnie od platformy sprzętowej klienta}
\boxy{Baza danych MySQL}{Zasób danych powinien być przechowywany na łatwej w obsłudze i bezpłatnej dystrybucji bazy danych, takiej jak np. \textbf{MySQL}}
\boxy{Dane o twórcach jak i projekcie}{Aplikacja będzie zawierała informacje o okolicznościach w jakich powstała i dla jakich celów}
\boxy{Model MVC}{Aplikacja będzie opierać się na modelu MVC}
\boxy{Dokumentacja}{Przebieg procesu powstawania jak i opis funkcjonalności i sposób wykorzystania poszczególnych funkcji będzie zawarty w dokumentacji projektu}