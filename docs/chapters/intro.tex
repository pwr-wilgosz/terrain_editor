\chapter{Cel i zakres projektu}
\section{Cel projektu}
Celem projektu jest wykonanie aplikacji serwerowej w technologii Ruby on Rails, pozwalającej na zarządzanie mapami 3d budowanymi na podstawie plików CSV.

Sam sposób wdrożenia programu jak i jego zadania wraz z implementacją przebiegały wedle przedstawionego pomysłu, który został zaakceptowany przez prowadzącego kurs.
Projekt miał na celu zapoznać studentów specjalizacji \textit{Inżynieria Internetowa} z procesem
budowania oprogramowania obecnie stosowanego w świecie aplikacji serwerowych.
Projekt dodatkowo dzięki wybranemu tematowi pomógł zgłębić technologie strony klienckiej
takie jak \textbf{JavaScript}.

\section{Zakres projektu}
Niniejszy projekt zawiera opis użytkowy i techniczny aplikacji oraz wykorzystanych narzędzi niezbędnych do stworzenia aplikacji w środowisku \textbf{Ruby on Rails} .

Proces tworzenia oprogramowania został podzielony na następujące etapy:

\begin{enumerate}
  \item{Analiza i Specyfikacja}
  \item{Projektowanie}
  \item{Implementacja}
  \item{Testy}
  \item{Ocena i Optymalizacja}
\end{enumerate}

\chapter{Analiza i Specyfikacja}
\section{Opis słowny zadania}
Celem pracy projektowej jest stworzenie aplikacji zdolnej do tworzenia i zarządzania mapami 3d terenu.
Niniejsza aplikacja serwerowa ma za zadanie wspomagać tworzenia ciekawych wizualizacji terenu dla osób zajmujących się hobbystycznie jak i zawodowo kartografią.
Docelowo tworzone przez użytkowników prace można będzie łączyć, zlecać wykonanie na nich danych obliczeń w celu wykorzystania ich w danym problemie.
Nasz serwis z aplikacją nazwałem \textbf{\textit{"3dMap"}}, ze względu na krótką i intuicyjną nazwę. Aplikację w przyszłości będzie można rozwinąć w celu szerszych zastosowań.

\section{Specyfikacja wymagań funkcjonalnych}
\begin{enumerate}
  \item \textbf{Wysyłanie emaili kontaktowych} - użytkownicy powinni mieć możliwość wysyłania maili do administratorów w razie problemów.

  \item \textbf{Prezentacja bazy punktów w formie mapy} - Jest to główna funkcja tej aplikacji, ma ona za zadnie z zadanego zbioru danych generować podgląd danego terenu w formie obrotowej mapki w przestrzeni trójwymiarowej.

  \item \textbf{Oddzielne sesje dla każdego użytkownika} - Ponieważ aplikacja będzie zawierała w sobie proces tworzenia jakiegoś elementu terenu niezbędne będzie wyodrębnienie pojedynczych działań aplikacji w formie sesji użytkowników, którzy swoje gotowe mapy będą mogli dzięki powiązaniu z kontem przechowywać w zdalnej przestrzeni dyskowej
  
  \item \textbf{Wczytywanie map z pliku} - Poza wczytywaniem map z bazy serwera możliwe będzie wczytywanie wcześniej zapisanych map z pliku. Funkcjonalność taka będzie przydatna gdy użytkownik po wykasowaniu mapy bądź usunięciu konta, chciałby odtworzyć swoje prace.
  
  \item \textbf{Łączenie kilku prac w jedną} - Serwer będzie mógł według kilku definiowanych zasad łączyć kilka zasobów w jedną wspólną mapę, taka funkcjonalność będzie przydatna przy tworzeniu pracy opartej na wkładzie kilku użytkowników
  
  \item \textbf{Generacja map} - Generacja map na podstawie już istniejącej z podanymi zasadami zmiany oraz tworzenie totalnie losowej mapy
  
  \item \textbf{Zapisywanie prac na serwerze} - Każda stworzona mapa będzie dostępna niezależnie od miejsca i sprzętu użytkownika poprzez interfejs webowy dostępny w przeglądarce internetowej
  
  \item{Eksport gotowych prac do określonych formatów}{Każdą z gotowych prac będzie można wyeksportować do formatu możliwego do użytku w celu prezentacji/wizualizacji z założenia są to formaty \textbf{*.pdf *.jpeg}.}
\end{enumerate}

\section{Specyfikacja wymagań niefunkcjonalnych}

\begin{enumerate}
  \item \textbf{Przejrzysty interfejs webowy} - Wymaganiem jest by aplikacja była nie przeładowana dodatkami i prosta w obsłudze dla osób nie posiadających zdolności programistycznych

  \item \textbf{Szybkość i oszczędność łącza} - Aplikacja powinna większość pracy wykonywać bez potrzeby generowania zbędnego ruchu sieciowego co pomoże zaoszczędzić zasoby klienta

  \item \textbf{Multiplatformowość} - Aplikacja powinna generować taki sam rezultat niezależnie od platformy sprzętowej klienta

  \item \textbf{Baza danych MySQL} - Zasób danych powinien być przechowywany na łatwej w obsłudze i bezpłatnej dystrybucji bazy danych, takiej jak np. \textbf{MySQL}
  
  \item \textbf{Dane o twórcach jak i projekcie} - Aplikacja będzie zawierała informacje o okolicznościach w jakich powstała i dla jakich celów
  
  \item \textbf{Model MVC} - Aplikacja będzie opierać się na modelu MVC
  
  \item \textbf{Dokumentacja} - Przebieg procesu powstawania jak i opis funkcjonalności i sposób wykorzystania poszczególnych funkcji będzie zawarty w dokumentacji projektu
\end{enumerate}
